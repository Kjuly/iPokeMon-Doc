%        File: ipokemon_paper.tex
%     Created: Fri May 11 17:00 PM 2012 C
% Last Change: Fri May 11 17:00 PM 2012 C
%
\documentclass{article}
\usepackage{CJKutf8}

% Package & settings for graphic
\usepackage[pdftex]{graphicx}
\usepackage{subfig} % Enable sub figure
\graphicspath{./figure/}
\DeclareGraphicsExtensions{.png,.jpg,.jpeg,.pdf}

% Package for References & Cite
\usepackage{natbib}

\title{基于位置服务的口袋妖怪类游戏开发}
\author{俞凯杰}


\begin{document}
\begin{CJK}{UTF8}{gbsn}
	% Make title
  \maketitle

	% Rename
  \renewcommand{\abstractname}{摘要}
	\renewcommand{\figurename}{图}
	\renewcommand{\refname}{参考文献}

	% References style & cite settings
	\bibliographystyle{unsrtnat}
	\setcitestyle{super, square, aysep={}, yysep={;}}

	% Begin content
  \begin{abstract}
  \end{abstract}

  \newpage
  \section{绪论}
	\subsection{概述}
	\subsection{课题背景及意义}
	\subsection{国内外研究发展现状}
	\subsubsection{LBS的研究发展现状}
	\subsubsection{iOS的发展现状}
	\subsection{本文的主要工作}
	\subsection{本文的组织结构}
	\subsection{本章小结}

	\section{方法与技术}
	\subsection{客户端设计模式}
	\subsubsection{MVC设计模式}
	\subsubsection{单例设计模式}
	\subsection{客户端开发框架:Cocoa}
	\subsection{服务器端开发框架:Bottle}
	\subsection{RESTful Web服务}
	\subsection{弹性云存储服务:Amazon EC2}
	\subsection{键值存储服务:Redis}
	\subsubsection{RDB快照}
	\subsubsection{AOF日志}
	\subsection{反向地理编码技术}
	\subsection{开发环境}
	\subsection{主要开发语言}
	\subsection{本章小结}

	\section{游戏总体设计}
	\subsection{设计概要}
	\subsubsection{游戏类别}
	\subsubsection{游戏面向对象}
	\subsection{整体框架}
	\subsection{主要功能结构}
	\subsection{游戏运行流程}
	\subsection{数据库设计}
	\subsubsection{客户端数据库设计}
	\subsubsection{服务器端数据库设计}
	\subsection{本章小结}

	\section{游戏详细设计}
	\subsection{项目开发规范}
	\subsubsection{系统目录规划}
	\subsubsection{命名规则}
	\subsection{客户端设计与实现}
	\subsubsection{视图层次关系}
	\subsubsection{数据模型}
	\subsubsection{单例}
	\subsection{服务端接口设计与实现}
	\subsection{基于位置服务的口袋妖怪分布机制实现}
	\subsection{本章小结}

	\section{游戏实现}
	\subsection{客户端界面}
	\subsection{基本功能}
	\subsection{操作说明}
	\subsection{本章小结}

	\section{总结与展望}
	\subsection{总结}
	\subsection{展望}


	% Import references datas
  %\bibliography{references.bib}

	\end{CJK}
\end{document}
