%        File: opening-report.tex
%     Created: Sun Jan 17 14:30 PM 2012 C
% Last Change: Sun Jan 17 14:30 PM 2012 C
%
\documentclass{article}
\usepackage{CJKutf8}

% Package & settings for graphic
\usepackage[pdftex]{graphicx}
\usepackage{subfig} % Enable sub figure
\graphicspath{./figure/}
\DeclareGraphicsExtensions{.png,.jpg,.jpeg,.pdf}

% Package for References & Cite
\usepackage{natbib}

\title{开题报告:基于位置服务的口袋妖怪类游戏开发}
\author{俞凯杰}


\begin{document}
\begin{CJK}{UTF8}{gbsn}
	% Make title
  \maketitle

	% Rename
  \renewcommand{\abstractname}{摘要}
	\renewcommand{\figurename}{图}
	\renewcommand{\refname}{参考文献}

	% References style & cite settings
	\bibliographystyle{unsrtnat}
	\setcitestyle{super, square, aysep={}, yysep={;}}

	% Begin content
  \section{选题的背景与意义}
  \subsection{项目开发的目的}

  随着移动通信和卫星定位技术的快速发展,LBS(Location Based Service,基于位置服务)技术已经受到人们的普遍关注,形形色色的采用LBS技术的应用也不断受到人们的厚爱。LBS是通过移动运营商的无线电通讯网络(如GSM网、CDMA网)或外部定位方式(如GPS)获取移动终端用户的位置信息(地理经度、纬度座标,甚至包括海拔高度),在GIS(Geographic Information System,地理信息系统)平台的支持下,为用户提供相应服务的一种增值业务。

  LBS可以被应用于生活、工作、航行等不同领域,它可以用于辨认一个人或物的位置,例如发现最近的餐厅,或者是朋友、同事当前的位置,也能通过用户目前所在的位置提供直接的手机广告业务,同时也包括个性化定制的天气预报业务,甚至提供本地化的社交游戏。

  与此同时,移动智能手机的普及和平民化,刺激了移动游戏市场的发展,国内外纷纷兴起一股移动游戏应用开发的潮流。这也可能预示着PC时代向移动终端设备时代的过渡时期的到来。

  目前市场上基于LBS技术的社交类(如Foursquare)、服务类(如Google地图导航)应用已浮现出许多,然而休闲游戏类应用仍然非常稀少,如果采用已有的经典游戏模型,那么完全可以在短时间内开发出一款采用LBS技术为核心的移动休闲游戏,吸引大量移动设备用户。
  
  而本项目正拟采用“口袋妖怪”这款风靡全球的经典角色扮演类游戏,将这款经典游戏和LBS技术相结合,把现实世界中的地理位置与游戏虚拟世界联系起来,用户(游戏玩家)就可以在现实生活场景的各个地方发现、捕捉口袋妖怪,并对其进行训练以达到可以和其他玩家相对抗的程度。该项目可以为全球的口袋妖怪迷提供一个新奇的玩法,让他们一起重温这款经典游戏;另一方面,还可以让用户外出边旅游边游戏,而不是每天宅在家里,从而达到很好的休闲娱乐效果。
  

  \subsection{国内外研究发展现状}

  随着移动通信技术和卫星导航技术的迅猛发展和广泛使用,逐渐产生了基于位置服务的需求,而轻便小巧的移动智能手机的普及则大力推动了LBS技术的发展。LBS通过全球卫星导航系统(Global Navigation Satellite System,简称GNSS)、GIS和无线通信技术,获取用户当前所在位置,并为用户提供个性化的服务。LBS技术涉及地理信息系统(Geographical Information Systems,简称GIS)、全球定位系统(Global Positioning Systems,简称GPS)、无线电频率识别,以及其它各种位置传感技术,具有不同程度的准确性、覆盖面和安装、维护成本\cite{L02}。
  
  LBS技术为当今社会提供了高效管理和持久控制的工具,该技术正被越来越多地应用于企业以及人们的日常生活中,以实现更好的目标\cite{L01}。它方便人们随时随地查找所处位置周边的有用信息,比如道路交通情况、附近口碑良好的餐厅的推荐等。LBS技术已经在社交(以Foursquare为例的签到服务)、生活(以大众点评网为例的餐饮服务)、工作(以Google地图为例的实时地图查询服务)等方面发挥了重要作用。
  
  虽然,LBS技术早在上个世纪就已经出现,但由于当时国内移动通信的带宽很窄、GPS的普及率比较低,最重要的是市场需求并不旺盛,所以,几家大的运营商虽然热情很高,但是整个市场并没有像预期的那样顺利启动,在一个很长的时间内,都是无人问津。随着3G的普及和流行,我国的LBS服务将会越来越完善,目前国内已经有一些厂商开始研发相关终端产品。比如聪聪科技的发现者系列,结合自身搭建的系统平台,可以实现对终端的精确定位,和历史轨迹查询等功能。部分企业也把握了这个时机,纷纷开发出了一些对于国内还很新鲜的应用,比如大众点评网、KK觅友等一些采用LBS技术的移动应用,最后从用户的使用和反馈情况可看出,反响还是很好的。相信LBS在中国将会在二三年内迎来一个爆发期。

  此外,也有不少研究人员正不断对LBS技术和系统框架进行研究,比如刘丹、彭黎辉利用基于空间信息网络(spatial information grid, SIG)的分布式思想对空间位置服务平台的架构进行了模型设计和实现,完成了地图数据和定位查询的分布式存储和检索\cite{L08};蒋郁、刘伟平等人在研究LBS平台的设计过程中,参考J2EE结构模型,并采用了组件式的设计方式\cite{L06};邹永贵、王剑提出了将SOA(Service-oriented Architecture,面向服务架构)架构运用于LBS,结合中间件技术,运用开放的Web服务技术,较好地解决了用户访问方式多样、异构兼容性不足和开发维护不易等问题\cite{L09};等等。可见,国内的LBS技术正越来越受关注。

  在国外,采用LBS技术的网络程序应用早已普及,特别像Foursquare这类网络社交应用,其成功有多方面的原因,它的“签到+勋章+市长”的模式刺激了许多用户“签到(Check in)”的兴趣,其实这个模式也可以认为是一种基于地理位置的游戏,只不过这种游戏是相对较轻的一种游戏,而像类似于《My Town》则是一种基于“地理位置+小游戏”的一种模式,加入了更多的游戏元素,比如“签到+房产买卖”的模式,并且还引入了道具的游戏元素,由于拥有更多的游戏LBS相关元素,这类基于地理的游戏很有可能发展出一些不同于Foursquare的盈利方式。国内的16Fun也是这方面的实践者。基于地理位置的游戏(Location Based Game,简称:LBG)应该是又一个值得被关注的地理位置领域的方向。

  为了满足在新兴的无线互联网市场中为用户提供挖掘丰富的空间内容,以及促进定位应用服务的发展这些需求,OGC(Open GIS Consortium,开放地理信息联盟)提出并已形成开放位置服务(Open Location Services,简称OpenLS)的倡议。OpenLS的愿景是提供开方式的接口,以实现互操作性,并且让任何设备随时随地都可在周边的服务热点中提供可操作、多任务、分布式、增值的定位服务和内容成为可能\cite{L13}。

  同时,苹果公司为其移动设备(如iPhono、iPad、iPod Touch)应用的开发提供的iOS SDK,免费面向所有开发人员,开发人员可以随时下载最新开发工具包进行应用的开发,这为这个深受广大用户喜爱的移动设备的应用开发带来了便捷,于此同时,其自带的CoreLocation框架更推动了LBS技术的发展。

	\section{项目开发的目标、基本内容和拟解决的技术难点}
  \subsection{项目开发目标}

  鉴于国内外LBS技术的研究和发展现状,在采用以LBS为核心技术的基础上,实现口袋妖怪类游戏的服务端和客户端的设计与实现。通过将现实世界中的地理位置与游戏虚拟世界相结合,开发基于位置服务的口袋妖怪类游戏,本项目具体开发目标如下:
  
  \begin{enumerate}
    \item 在客户端,完成支持移动设备(先支持iPhone)的游戏开发;
    \item 在服务端,搭建高效的LBS系统,包括Web服务和LDAP(Lightweight Directory Access Protocol轻量级目录访问协议)服务;
    \item 对不同地域(先对杭州地区进行初期试验)的口袋妖怪分布进行分配,以及确定其出现的概率的计算算法,以达到公平和可玩性。
  \end{enumerate}
  
  \subsection{项目的基本内容}
  \subsubsection{客户端}

  为客户端开发一款口袋妖怪类游戏,用户可以通过自己的移动设备从网上下载从而免费获得游戏包,安装即可使用。

  由于任天堂的“口袋妖怪”通常运行于自己生产的游戏设备(如Game Boy),或者是PC模拟器上,游戏源代码并不向外界公开,如果需要结合这款游戏,则必须重新开发。但是对本项目来说,最重要的是,它的基本游戏模式是公开的:它是一款角色扮演类游戏,通过控制角色在不同城市之间旅行,发现和捕捉新奇的口袋妖怪,对抓到的口袋妖怪进行训练,从而与其他训练师进行比赛。另外,口袋妖怪还有一套令人兴奋的进化系统,即口袋妖怪训练等级达到一定程度后,便会自动进化成其它形态的口袋妖怪。

  而本项目把LBS技术应用于该游戏,对该游戏进行一些改进:
  
  \begin{enumerate}
    \item 将原先的角色扮演对象从游戏小人替换成用户真实的自己;
    \item 通过一些现有服务(如Google Maps API),将真实世界中的地图映射到游戏中,用户所在位置即为游戏中所在位置;
    \item 将口袋妖怪的分布从现实世界中通过一定的算法映射到游戏中,用户设备一旦检测到身边的口袋妖怪,则提醒用户,并进行战斗(战斗目的可以有:捕捉出现的口袋妖怪、训练自己的口袋妖怪,以及为了完成某些任务);
    \item 从原先的单机游戏改进到多人模式的网络游戏,用户可以和身边的其它玩家对抗,增加乐趣。
  \end{enumerate}

  \subsubsection{服务端}

  服务端主要提供基于位置服务和数据的处理任务。

  口袋妖怪在现实世界中的分布并不是一成不变的,它通过服务器采用算法每隔一定时间自动生成口袋妖怪分布图,映射到现实中。客户端(用户移动设备)通过一定时间间隔发送消息到服务端,询问是否有口袋妖怪。另外,用户捕捉到的口袋妖怪数据将存储于服务端的数据库,用户只能通过验证获取自己的数据信息。服务端需要实现的基本功能如下:

  \begin{enumerate}
    \item 自动生成口袋妖怪分布图;
    \item 判断用户所在位置是否有口袋妖怪和其他训练师;
    \item 用户身份验证,以及处理用户在数据库中的数据。
  \end{enumerate}

  考虑到项目开发周期,以下几个功能暂时不纳入项目开发计划,但后期需要实现:
  \begin{enumerate}
    \item 用户位置信息隐私问题;
    \item 游戏社交元素的引入,即用户可以加认识的玩家为好友,可以一起执行任务,或者互相赠送刚捕获的口袋妖怪等。
  \end{enumerate}


  \subsection{拟解决的技术难点}
  \begin{enumerate}
    \item LBS系统的搭建;
    \item 客户端游戏的设计与开发;
    \item 游戏中口袋妖怪的具体分布算法设计;
    \item 客户端和服务端之间的高效通信问题。
  \end{enumerate}

  \section{项目开发的方法、技术路线和步骤}
  \subsection{项目开发模式}

  本项目开发采用MVC(Model-View-Controller,模型-视图-控制器)的基本开发模式,将展示层和逻辑层分离,以方便项目管理和后期的游戏逻辑修改。

  MVC模式是软件工程中的一种软件架构模式,把软件系统分为三个基本部分:模型(Model)、视图(View)和控制器(Controller)。MVC模式的目的是实现一种动态的程式设计,使后续对程序的修改和扩展简化,并且使程序某一部分的重复利用成为可能。除此之外,此模式通过对复杂度的简化,使程序结构更加直观。软件系统通过对自身基本部份分离的同时也赋予了各个基本部分应有的功能。MVC模式各模块功能如下:
  \begin{enumerate}
    \item 视图(View):界面设计人员进行图形界面设计。
    \item 模型(Model):程序员编写程序应有的功能(实现算法等等)、数据库专家进行数据管理和数据库设计(可以实现具体的功能)。
    \item 控制器(Controller):负责转发请求,对请求进行处理。
  \end{enumerate}

  \subsection{系统架构}

  系统采用C/S(Client/Server,客户端/服务器)模式进行架构。

  C/S模式又称C/S结构,是20世纪80年代末逐步成长起来的一种模式,是软件系统体系结构的一种。C/S模式的关键在于功能的分布,一些功能放在客户端上执行,另一些功能放在服务端上执行。功能的分布在于减少计算机系统的各种瓶颈问题。C/S模式简单地讲就是基于企业内部网络的应用系统,与B/S(Browser/Server,浏览器/服务器)模式相比,C/S模式的应用系统最大的好处是不依赖外网环境,即无论用户是否能够上网,都不影响应用,对于本项目来说,用户可以随时随地进行游戏,查看自己的口袋妖怪情况,但是在无网络的情况下无法获取周边训练师的存在情况。

  \subsection{服务端系统平台}

  服务端系统平台拟采用Nginx进行服务器的配置和实现。

  Nginx ("engine x") 是一个高性能的HTTP和反向代理服务器,也是一个IMAP/POP3/SMTP代理服务器。Nginx是由Igor Sysoev为俄罗斯访问量第二的Rambler.ru站点开发的,它已经在该站点运行超过四年多了。Igor将源代码以类BSD许可证的形式发布。自Nginx发布四年来,Nginx已经因为它的稳定性、丰富的功能集、示例配置文件和低系统资源的消耗而闻名了。目前国内各大门户网站已经部署了Nginx,如新浪、网易、腾讯等;国内几个重要的视频分享网站也部署了Nginx,如六房间、酷6等。新近发现Nginx技术在国内日趋火热,越来越多的网站开始部署Nginx。Nginx作为一个高性能Web和反向代理服务器,具备优秀的特性如下\cite{WIKI_Nginx}:

  \begin{enumerate}
    \item 在高连接并发的情况下,Nginx是Apache服务器不错的替代品:Nginx在美国是做虚拟主机生意的老板们经常选择的软件平台之一。能够支持高达50000个并发连接数的响应,同时,Nginx为我们选择了epoll and kqueue作为开发模型;
    \item Nginx作为负载均衡服务器:Nginx既可以在内部直接支持Rails和PHP程序对外进行服务,也可以支持作为HTTP代理服务器对外进行服务。Nginx采用C进行编写,不论是系统资源开销还是CPU使用效率都比Perlbal要好很多;
    \item 作为邮件代理服务器:Nginx同时也是一个非常优秀的邮件代理服务器(最早开发这个产品的目的之一也是作为邮件代理服务器);
    \item Nginx是一个安装非常的简单、配置文件非常简洁(还能够支持perl语法)、Bugs非常少的服务器:Nginx启动特别容易,并且几乎可以做到7x24不间断运行,即使运行数个月也不需要重新启动。还能够在不间断服务的情况下进行软件版本的升级。
  \end{enumerate}

  \subsection{客户端平台和采用的框架}

  客户端平台拟采用iOS平台,采用Cocoa框架进行游戏的开发。

  iOS平台已经极大地改变了新一代移动游戏的前景,它的独特性、连接性、个人集成性,流行以及新颖的界面,吸引了无数开发者。iPhone独特的用户界面已经派生了一些全新类别的游戏。改变设备倾斜的角度可以控制倾斜类的游戏,例如Lima Sky开发的Doodle Jump和NimbleBit开发的Scoops;多点触摸的游戏(例如Igloo Games开发的Bed Bugs)可以使得用户专注于游戏,因为它迫使用户同时操作多个游戏对象;Smule开发的娱乐应用程序Ocarina和Leaf Trombone允许用户通过向iPhone的麦克风吹气来弹唱虚拟乐器\cite{B02}。

  Cocoa框架是苹果公司为Mac OS X(包括优化了的适用于移动设备的iOS系统)所创建的原生面向对象的编程环境,是Mac OS X上五大API之一。Cocoa框架包含开发Mac OS X所需的类库、API和运行环境。通过Cocoa框架,开发人员可以以开发Mac OS X的方式进行项目的开发,而且开发出来的应用将自动继承Mac OS X的基本功能并遵循苹果公司的人机界面准则,而且还可以直接可以访问功能强劲的Unix系统底层\cite{MAN_Cocoa}。此外,Cocoa框架还遵循MVC的设计模式。

  \subsection{编程语言}

  服务端,拟采用Python和HTML+CSS分别进行服务端逻辑和界面(主要为管理员提供)的编写。
  客户端,由于采用Cocoa框架,所以采用Objective C语言进行游戏的开发。

  \subsection{开发工具}

  采用Vim和XCode 4.1进行项目的开发。

  Vim是从vi发展出来的一个文本编辑器。代码补完、编译及错误跳转等方便编程的功能特别丰富,在程序员中被广泛使用,与Emacs并列成为类Unix系统用户最喜欢的编辑器。

  XCode它是苹果公司向开发人员提供的IDE(Integrated development environment,集成开发环境),用于开发Mac OS X的应用程序。有优秀的代码自动补全功能,同时集成LLVM编译器进行项目的编译。

  \subsection{LBS技术实现}

  通过iOS SDK自带的CoreLocation框架,对移动终端进行定位;采用Google Maps API把现实世界的地图映射到游戏中。

  苹果公司为iOS平台的开发提供了开发工具包:iOS SDK中含有一个CoreLocation框架库。使用CoreLocation框架,可以确定设备当前的经度和纬度,从而提供位置相关的设置和事件。该框架通过硬件设备获取附近的信号,从而定位用户的位置\cite{iOSLIB}。这大大降低了LBS应用开发者的负担,也推动了LBS技术的发展。

  Google Maps API是Google为开发者提供的Maps编程API。它允许开发者在不必建立自己的地图服务器的情况下,将Google Maps地图数据嵌入到网站之中,从而实现嵌入Google Maps的地图服务应用,并借助Google Maps的地图数据为用户提供位置服务。值得一提的是,Google Maps API最大的优势,在于它是一个开放的系统,即用户完全可自定义非常的内容,从功能的地图、控件、事件,到专业的地图坐标系、地图类型、周边搜索等,用户通过Google Maps API均可以自定义。

  Google在推出地图、卫星图之后,又推出了地形图,它可以让用户在Google Maps上看到地图的三维地形。这是一个非常有用的功能,新的地形模式主要侧重于地图的物理特性,如高山、峡谷、植被等。地形模式甚至可以包含非常小的山、山路,并且使用只有细小差别的明暗来让人们更好地从卫星图片上理解地理高度的变化。\cite{B01}。对

  \subsection{游戏中口袋妖怪分布设计}

  游戏中口袋妖怪的分布需要考虑经度、纬度和海拔高度三个基本因子,另外,也应该尽量考虑地形的影响,比如草丛中应该出现草系口袋妖怪,湖泊上(用户在船上的可能)应该出现水系口袋妖怪,等等。而这不能够单单从一个算法可以很好解决的,所以需要结合基本出现概率的算法和实地考察(或通过Google Maps查询)对特殊区域进行特殊对待。如何设计口袋妖怪的分布是项目后期需要着重解决的问题。

  \section{项目工作总体安排与时间进度}
  2011.11.30-2011.12.20:了解课题相关内容,查找中、英文资料
  2011.12.21-2011.01.21:查阅文献资料,完成外文翻译、文献综述和开题报告
  2012.02.01-2012.02.20:学习项目相关技术:iOS SDK、Google Maps API等
  2012.02.21-2012.02.29:购买、配置服务器
  2012.03.01-2012.03.20:实现服务端基本功能,设计数据库
  2012.03.21-2012.04.30:游戏客户端开发
  2012.05.01-2012.05.08:进行游戏测试
  2012.05.09-2012.05.21:服务端优化
  2012.05.21-2012.05.31:整理资料、完成毕业论文
  2012.06.01-2012.06.10:上交毕业论文、准备毕业设计答辩;继续测试、完善系统

	% Import references datas
	\bibliography{references.bib}

	\end{CJK}
\end{document}
  Google在推出地图、卫星图之后,又推出了地形图,它可以让用户在Google Maps上看到地图的三维地形。这是一个非常有用的功能,新的地形模式主要侧重于地图的物理特性,如高山、峡谷、植被等。地形模式甚至可以包含非常小的山、山路,并且使用只有细小差别的明暗来让人们更好地从卫星图片上理解地理高度的变化。\cite{B01}
